\chapter{Specifikacija programske potpore}
		
	\section{Funkcionalni zahtjevi}
			
			\textbf{\textit{dio 1. revizije}}\\
			
			\textit{Navesti \textbf{dionike} koji imaju \textbf{interes u ovom sustavu} ili  \textbf{su nositelji odgovornosti}. To su prije svega korisnici, ali i administratori sustava, naručitelji, razvojni tim.}\\
				
			\textit{Navesti \textbf{aktore} koji izravno \textbf{koriste} ili \textbf{komuniciraju sa sustavom}. Oni mogu imati inicijatorsku ulogu, tj. započinju određene procese u sustavu ili samo sudioničku ulogu, tj. obavljaju određeni posao. Za svakog aktora navesti funkcionalne zahtjeve koji se na njega odnose.}\\
			
			
			\noindent \textbf{Dionici:}
			
			\begin{packed_enum}
				
				\item Dionik 1
				\item Dionik 2				
				\item ...
				
			\end{packed_enum}
			
			\noindent \textbf{Aktori i njihovi funkcionalni zahtjevi:}
			
			
			\begin{packed_enum}
				\item  \underbar{Aktor 1 (inicijator) može:}
				
				\begin{packed_enum}
					
					\item funkcionalnost 1
					\item funkcionalnost 2
					\begin{packed_enum}
						
						\item  podfunkcionalnost 1 
						\item  podfunkcionalnost 2
				
					\end{packed_enum}
					\item  funkcionalnost 3
					
				\end{packed_enum}
			
				\item  \underbar{Aktor 2 (sudionik) može:}
				
				\begin{packed_enum}
					
					\item funkcionalnost 1
					\item funkcionalnost 2
					
				\end{packed_enum}
			\end{packed_enum}
			
			\eject 
			
			
				
			\subsection{Obrasci uporabe}
				
				\textbf{\textit{dio 1. revizije}}
				
				\subsubsection{Opis obrazaca uporabe}
					\textit{Funkcionalne zahtjeve razraditi u obliku obrazaca uporabe. Svaki obrazac je potrebno razraditi prema donjem predlošku. Ukoliko u nekom koraku može doći do odstupanja, potrebno je to odstupanje opisati i po mogućnosti ponuditi rješenje kojim bi se tijek obrasca vratio na osnovni tijek.}\\
					

					\noindent \underbar{\textbf{UC$<$broj obrasca$>$ -$<$ime obrasca$>$}}
					\begin{packed_item}
	
						\item \textbf{Glavni sudionik: }$<$sudionik$>$
						\item  \textbf{Cilj:} $<$cilj$>$
						\item  \textbf{Sudionici:} $<$sudionici$>$
						\item  \textbf{Preduvjet:} $<$preduvjet$>$
						\item  \textbf{Opis osnovnog tijeka:}
						
						\item[] \begin{packed_enum}
	
							\item $<$opis korak jedan$>$
							\item $<$opis korak dva$>$
							\item $<$opis korak tri$>$
							\item $<$opis korak četiri$>$
							\item $<$opis korak pet$>$
						\end{packed_enum}
						
						\item  \textbf{Opis mogućih odstupanja:}
						
						\item[] \begin{packed_item}
	
							\item[2.a] $<$opis mogućeg scenarija odstupanja u koraku 2$>$
							\item[] \begin{packed_enum}
								
								\item $<$opis rješenja mogućeg scenarija korak 1$>$
								\item $<$opis rješenja mogućeg scenarija korak 2$>$
								
							\end{packed_enum}
							\item[2.b] $<$opis mogućeg scenarija odstupanja u koraku 2$>$
							\item[3.a] $<$opis mogućeg scenarija odstupanja  u koraku 3$>$
							
						\end{packed_item}
					\end{packed_item}
				
					
					\subsubsection{Dijagrami obrazaca uporabe}
					
					\begin{figure}[H]
						\includegraphics[scale=0.30]{dijagrami/dijagram1.jpg} 
						\centering
						\caption{Funkiconalnosti učenja i upravljanja riječima}
						\label{fig:dijagram1}
					\end{figure}
					
					\begin{figure}[H]
						\includegraphics[scale=0.35]{dijagrami/dijagram2.jpg} 
						\centering
						\caption{Funkcionalnosti upravljanja računima}
						\label{fig:dijagram2}
					\end{figure}
					
					\begin{figure}[H]
						\includegraphics[scale=0.34]{dijagrami/dijagram3.jpg} 
						\centering
						\caption{Funkcionalnosti upravljanja jezicima i rječnicima}
						\label{fig:dijagram3}
					\end{figure}	
				
			\subsection{Sekvencijski dijagrami}
				
				\textbf{\textit{dio 1. revizije}}\\
				
				\textit{Nacrtati sekvencijske dijagrame koji modeliraju najvažnije dijelove sustava (max. 4 dijagrama). Ukoliko postoji nedoumica oko odabira, razjasniti s asistentom. Uz svaki dijagram napisati detaljni opis dijagrama.}
				\eject
	
				\section{Ostali zahtjevi}

				\begin{itemize}
					\item sustav korisniku daje povratnu informaciju o točnosti njegova odgovora
					\item sustav u razumnom vremenu prezentira riječi nakon odabira načina rada
					\item sustav mora imati potporu hrvatskih dijakritičih znakova
					\item sustavu iz javne mreže pristupamo protokolom HTTPS
					\item sustav je dovoljno jednostavan i intuitivan za bilo koju dobnu skupinu korisnika 
					\item za korištenje sustava korisniku je potrebno poznavanje hrvatskog jezika
					\item rječnik mora imati barem 10 riječi
					\item prijevodi rječi moraju biti ispravni
				\end{itemize}
			 
			 
			 
	