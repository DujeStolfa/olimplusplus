\chapter{Dnevnik promjena dokumentacije}
		
		\textbf{\textit{Kontinuirano osvježavanje}}\\
				
		
		\begin{longtblr}[
				label=none
			]{
				width = \textwidth, 
				colspec={|X[2]|X[11]|X[5]|X[4]|}, 
				rowhead = 1
			}
			\hline
			\textbf{Rev.}	& \textbf{Opis promjene/dodatka} & \textbf{Autori} & \textbf{Datum}\\[3pt] \hline
			0.1 & Napravljen predložak.	& Duje Štolfa & 25.10.2023. 		\\[3pt] \hline 
			0.2	& Opis projektnog zadatka. & Gabrijel Čobanov \newline Karlo Kuzle & 28.10.2023. 	\\[3pt] \hline 
			0.3 & Razrađeni funkcionalni zahtjevi, \newline aktori i dionici. \newline Nabrojani \textit{Use Caseovi}. & Nina Bulić \newline Duje Štolfa & 29.10.2023. \\[3pt] \hline 
			0.4 & Opisani obrasci uporabe \newline i ostali zahtjevi. & Frane Kuzmanić \newline Gabijel Čobanov \newline Nikša Brala \newline Karlo Kuzle \newline Nina Bulić \newline Duje Štolfa \newline Ivo Žilić & 31.10.2023.\\[3pt] \hline 
			0.7 & Dijagrami obrazaca uporabe & Frane Kuzmanić & 31.11.2023. \\[3pt] \hline 
			0.6 & Dodani sekvencijski dijagrami \newline i njihovi opisi &  Nina Bulić \newline Duje Štolfa \newline Ivo Žilić & 2.11.2023. \\[3pt] \hline 
			0.7 & Dijagram baze podataka & Nikša Brala & 5.11.2023. \\[3pt] \hline
			0.8 & Opis baze i tablica baze podataka & Karlo Kuzle & 7.11.2023. \\[3pt] \hline 
			0.9 & Dodan opis arhitekture sustava & Duje Štolfa \newline Gabrijel Čobanov \newline Frane Kuzmanić & 10.11.2023. \\[3pt] \hline 
			0.10 & Dijagrami razreda & Nikša Brala & 11.11.2023. \\[3pt] \hline 
			0.11 & Opis dijagrama razreda & Gabrijel Čobanov \newline Nina Bulić \newline Ivo Žilić \newline Duje Štolfa & 13.11.2023. \\[3pt] \hline 
			\textbf{1.0} & Prva revizija dokumentacije & * & 17.11.2023. \\[3pt] \hline 
		\end{longtblr}
	
	
		\textit{Moraju postojati glavne revizije dokumenata 1.0 i 2.0 na kraju prvog i drugog ciklusa. Između tih revizija mogu postojati manje revizije već prema tome kako se dokument bude nadopunjavao. Očekuje se da nakon svake značajnije promjene (dodatka, izmjene, uklanjanja dijelova teksta i popratnih grafičkih sadržaja) dokumenta se to zabilježi kao revizija. Npr., revizije unutar prvog ciklusa će imati oznake 0.1, 0.2, …, 0.9, 0.10, 0.11.. sve do konačne revizije prvog ciklusa 1.0. U drugom ciklusu se nastavlja s revizijama 1.1, 1.2, itd.}