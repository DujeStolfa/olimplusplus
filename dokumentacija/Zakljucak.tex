\chapter{Zaključak i budući rad}

\par Razvoj projekta je bilo važno iskustvo za sve članove tima. Dobili smo uvid u alate za komunikaciju
ideja, kao što su stečene i vještine sinkronizacije sa drugim članovima. Naučeno je puno o konkretnom razvoju
aplikacija na webu, kao i o Reactu. Vrijeme razvoja je bilo definirano količinama znanja individualnih članova tima, 
te da ponovno pravimo isti projekt sa svime naučenim, išlo bi brže.
\par Puno vremena je bilo uloženo u idejnu razradu aplikacije, te smo se posvetili detaljima funkcionalnih
zahtjeva i razrade svih obrazaca uporabe prije nego što smo krenuli programirati, što je u konačnici uštedilo
puno vremena. Nismo imali dvoumljenja oko implementacije, samo smo mogli pisati po već razrađenim
obrascima. 
\par Za bržu i kvalitetniju izradu projekta bi samo bilo potrebno da svi članovi imaju istu količinu
znanja vezanu za odabrani radni okvir, kao i sličnu količinu iskustva u izradi web aplikacija. Puno stvari
u razvoju na webu se ponavlja, projekti imaju slične strukture datoteka, koriste se isti programski obrasci
kao rješenje dobro poznatih problema. Izrada baze podataka je donekle standardiziran postupak, 
povezivanje na istu i povlačenje podataka se može napraviti na više načina. Sve se svodi na odabir nekih od 
svih dostupnih tehnologija, te iskustvo s radom u istim. Korisničko sučelje se može napraviti relativno brzo
uz pomoć biblioteka i radnih okvira. Puno gotovih rješenja za web već postoji, vještina ih je znati spojiti
i znati kako te bibilioteke međusobno komuniciraju. 
\par Glede perspektiva za nastavak rada u projektnoj grupi, bilo kakav budući rad može biti samo lakši,
budući da je tim prošao kroz svoje ranije faze, te svi članovi lakše znaju komunicirati jedni s drugima i 
uskočiti tamo gdje treba. Svi su upoznati s vlastitim snagama i slabostima.
\par Sve tražene funkcionalnsoti projektnog zadatka su implementirane i razrađene. 


\eject